%# -*- coding: utf-8-unix -*-
%%==================================================
%% chapter01.tex for SJTU Master Thesis
%%==================================================

%\bibliographystyle{sjtu2}%[此处用于每章都生产参考文献]

\chapter{检测模型的应用}
\label{chap:appli}




\section{寻找水军行动规律}

本节我们将讨论如何使用SpamTracer模型在数据集上进行更多的探索。水军检测模型的本职工作是作为过滤系统,对每一条新发布的评论进行实时检测。该功能的前提要求是海量的用户数据和评论数据基础,是只有O2O商业平台自身才能够做到的。然而,普通用户也有想要分辨水军评论的需求,这可以改善普通用户的消费体验。普通用户不可能掌握大量用户数据和评论数据,也不懂如何操作监测模型,普通用户的需求是几条可以帮助他们快速识别水军评论的建言。因此,我们接下来的目标是利用SpamTracer对数据集进行分析,掌握水军行动的一些规律。这些规律可以在一定程度上帮助普通用户们分辨出水军评论。

网络上也有人总结过部分判断水军的经验规律。例如,在店铺在开店的初期会试图雇佣一些水军帮助造势,所以不能太信任早期的评论。还有,水军们会周期性地进行集团行动,等等。此外,还有一种言论称,水军们倾向于寻找竞争激烈的热门商圈的多家商铺进行合作,说服商铺们使用水军服务。作为处于激烈竞争中的商铺,如果周围的竞争对手都雇佣了水军的话,自己也很可能被迫使用水军服务。然而归根结底,上面说的这些都只是经验论,并没有具体的数据支持。但是现在拥有了SpamTracer和数据集的我们就可以用数据来判断这些经验论是否合理。

由于我们收集到的数据集是一个部分标签的数据集,拥有标签的数据量较少。为了增大数据量,我们首先要使用SpamTracer模型给数据集中未标签的数据打上标签。数据量越大,结果越准确。接下来我们将介绍本次毕业设计验证的三条水军行动规律:


\subsection{寻找日期与每日水军评论数量的关系}

通过寻找日期与每日水军评论数量的关系,我们可以获得水军们的周期性集团行动的规律。我们打算以月为周期和以周为周期进行统计。在扩充后的数据集中,我们将所有被SpamTracer判定为水军评论的评论时间按照月份和星期进行归类,再制作出图表观察规律即可。此外,为了使统计结果更加合理,我们过滤掉了拥有评论数量较少的店铺。这些数据会为计算带来误差。



\subsection{寻找开业天数与每日水军评论数量的关系}

通过寻找开业天数与每日水军评论数量的关系,我们可以知道店铺倾向于在开业的何时雇佣水军。在扩充后的数据集中,我们计算所有被SpamTracer判定为水军评论的评论日期与对应店铺开店日期的日期差,这个日期差就是该水军评论是在对应店铺的开业第几天发布的。我们统计所有的日期差,制作图表观察规律。此外,我们也可以顺便观察一下店铺们雇佣水军的倾向。我们统计了所有店铺的所有评论中水军评论的占比,并统计不同比例的区间内店铺的数量。我们打算观察现在的店铺中究竟有多少雇佣了大量的水军用户。为了使统计结果更加合理,我们过滤掉了拥有评论数量较少的店铺,以及开业时长较短的店铺。这些数据会为计算带来误差。



\subsection{寻找不同店铺间距与共有水军数量的关系}

寻找不同店铺间距与共有水军数量的关系的操作比较复杂。首先我们定义共有水军的概念:

\begin{defn}
	\textbf{共有水军}:同时为复数个店铺发布虚假评论的水军账号。
\end{defn}

为了探究是否存在水军与小商圈内多家店铺同时开展合作的现象,我们需要计算数据集内每两个店铺的相隔距离和共有水军的数量。如果这两个量存在一定的联系的话,我们可以绘制二者的关系图,观察规律。假设$\mathbf{M}$代表店铺数据集,函数$I(x|y)$代表符合条件$y$的值$x$的数量,那么$\mathbf{K}$代表共有水军的数量可以通过公式~\eqref{equation:share}计算:
\begin{equation}
\label{equation:share}
\begin{aligned}
\mathbf{K}_{ij} = I(y| &\forall y \in i.reviewer, \exists z \in j.reviewer, st. z.name = y.name, \\
&z.status = y.status = spammer, i \in \mathbf{M}, j \in \mathbf{M})\\
\end{aligned}
\end{equation}

然而,直接按照上述公式的方法进行计算效率很低。为了提高计算速度,并使得结果更加合理,我们设立了一个评论数量阈值来过滤掉那些评论数较低的店铺。这类店铺会干扰最后的统计结果。算法~\ref{alg:shared spammer}展示了改进后的具体计算步骤:


\IncMargin{1em}
\begin{algorithm2e}[H]
	\caption[计算每两个商铺的间距和共有水军数量]{Calculate the interval distance and amount of shared spammers between two shops}
	\label{alg:shared spammer}
	\SetAlgoNoLine
	\KwIn{
		Set of shops $M$,
		set of reviewers $R$,
		Threshold of shop review number $\delta$\;}
	
	\KwOut{Pairs of distances and amount of shared spammers $H(d, n)$}
	
	\BlankLine
	
	\For{$\forall m \in M$}{
		\lIf{m.numberOfReviews $<$ $\delta$}{\textbf{continue}}
		$R_m = \{r \in R|m.reviewer = r\}$\;
		Run SpamTracer on $R_m$ to get the class of every reviewer stored as $r.status$\;
		Add $m$ into set $M'$, $M'$ is a set stores all the useful shops\;
	}
	\For{$i=1,...,length(M')$}{
		\For{$j=1,...,i$}{
		$d=Distance(M'[i].reviewer, M'[j].reviewer)$\;
		$n=0$\;
		$List=M'[i].reviewer + M'[j].reviewer$\;
		Sort List by the name of reviewers\;
		\For{$k=1,...,length(List)-1$}{
			\lIf{List[k].name = List[k+1].name \&\& List[k].status = spammer \&\& List[k].shop != List[k+1].shop}{$n = n + 1$}
			}
		Add $(d,n)$ into set $H$\;
		}
	}
\end{algorithm2e}
\DecMargin{1em}


首先是准备阶段,算法过滤掉评论数低于$\delta$的店铺,按照店铺归类为该店铺写过评论的账号,再用SpamTracer对这些账号进行分类(1-6)。将该店铺及为其写过虚假评论的账户存起来。准备阶段结束后,算法对所有保存下来的店铺按对进行遍历,并计算每一对店铺的间距和共有水军数量(7-18)。算法的时间复杂度是$O(klog(k)n^2)$,其中$k$是店铺拥有的平均评论数,$n$是过滤后的店铺数量。$k$可以视为一个常量,故整体的时间复杂度是$O(n^2)$。

总体来说,在SpamTracer的帮助下我们可以对这三条水军行动规律进行分析并作出判断。在实验部分我们将展示这些行动规律的验证结果。







