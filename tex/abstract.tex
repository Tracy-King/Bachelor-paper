%# -*- coding: utf-8-unix -*-
%%==================================================
%% abstract.tex for SJTU Master Thesis
%%==================================================

\begin{abstract}

近年来,O2O(Online To Offline)电子商务平台在居民的消费生活中作用越来越重要。然而,在名誉和利益的驱动下,有一部分人试图使用网络诈骗的方式,例如发布水军评论和虚假内容,恶意操作线上市场。这种行为极大地危害了线上市场的环境,应当被立即侦测并驱除。而且,比起简单的僵尸网络机器水军,那些人工水军的隐蔽性和危害性更大。目前在网络水军检测领域已经存在各种成熟的方法。这些方法利用水军的爆发式活动特征和账户信息等特征进行机器学习,识别出水军用户。但是随着水军技术的不断革新,水军们已经掌握了躲避传统机器学习检测的方法,他们会控制水军活动的速率,并模仿正常用户的样子完善账户信息,传统的水军检测系统使用的算法和特征面临着效率不足的问题,我们急需找到与之前的水军检测原理相异的,高效的新特征和新方法。我们的调查显示地理位置因素可以很好地反映水军用户和普通用户之间的差异。在这篇论文中,我们分析了评论中店铺的地理位置,找到了水军用户和普通用户中地理位置特征分布的差异,并提出了一个基于隐马尔可夫模型的\emph{SpamTracer}模型用于分辨水军用户和正常用户。我们的实验结果展示了SpamTracer可以在不平衡的数据集中达到71\%的准确率(Accuracy)和76\%的召回率(Recall)。这个测试结果在稳定性上可以超过一些优秀的经典方法。此外,SpamTracer可以帮助我们分析人工水军的行动规律。这些规律反映了商家们倾向于在合适何地雇佣人工水军来增加收入。我们还发现了一小部分人工水军倾向于和在一个小商圈内的多个商家达成合作关系。

\keywords{\large O2O电子商务平台, 水军检测, 地理位置, 隐马尔可夫模型}
\end{abstract}

\begin{englishabstract}

Nowadays, O2O(Online To Offline) commercial platforms are playing a crucial role in our daily purchases. However, some people are trying to manipulate the online market maliciously by opinion spamming, a kind of web fraud behavior like writing fake reviews, due to fame and profits, which will harm online purchasing environment and should be detected and eliminated. Moreover, manual spammers are more deceptive compared with old web spambots. Several sophisticated and efficient methods have been proposed in spamming review detection field. Those methods apply spamming burst features and account characteristics to machine learning algorithms, and identify the spammers. However, with the evolution of spamming approaches, spammers can get away from detection systems. They can control the rate of spamming actions, and disguise as normal accounts. Thus current detection systems and features are facing with low efficiency. It's essential to find new efficient features and algorithms. Our investigation presented that geolocation can well reflect the distinctions between spammers and normal users. In this research, we analyzed the geolocations of shops in review, found the distinct distribution features of those in spamming and non-spamming users, and proposed a \emph{SpamTracer} model that can identify spammers and non-spammers by exploiting an improved HMM(Hidden Markov Model). Our experiment demonstrated that SpamTracer could achieve 71\% accuracy and 76\% recall in the unbalanced dataset, outperforming some excellent classical approaches in the aspect of stability. Furthermore, SpamTracer can help to analyze the regularities of spamming actions. Those regularities reflect the time and location in which online shops are likely to hire spammers to increase their turnover. We also found that a small group of spammers tend to work with plural shops located in a small business zone.

\englishkeywords{\large O2O Commercial Platform, Spamming Detection, Geolocation, Hidden Markov Model}
\end{englishabstract}

