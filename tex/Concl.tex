%# -*- coding: utf-8-unix -*-
%%==================================================
%% chapter01.tex for SJTU Master Thesis
%%==================================================

%\bibliographystyle{sjtu2}%[此处用于每章都生产参考文献]
\chapter{结论}
\label{chap:concl}


在这篇论文中,我们进行了利用地理位置特征检测O2O商业平台中水军的研究。这篇论文分七章介绍了我们的工作。在第一章绪论中,我们简单介绍了水军检测领域的问题背景,介绍了当前互联网产业的飞速发展,以及水军对于电商平台的巨大危害,阐述了我们的研究的重要性。之后我们介绍了水军检测领域的研究现状,介绍了众多前人的工作,并在前人工作的基础上提出了自己的想法——利用地理位置特征作为切入点,建立模型进行水军检测工作。在第二章前言部分,我们简单介绍了本文中用到的知识基础——机器学习相关理论、水军检测领域的分类模型。随机过程以及隐马尔可夫模型。机器学习方法是本篇论文中的主要思路,应用在水军检测领域的相关模型给我们提供思路,随机过程是我们研究的前提假设,而隐马尔可夫模型是我们模型的基础。在掌握了基础知识之后,我们在第三章详细地介绍了地理位置特征的提取。地理位置特征是我们的检测模型中至关重要的一环,选择怎么样的地理位置特征直接关系到检测模型的运行效果。我们提出了一个类似“半径”的地理位置特征,并用统计数据和分布数据支持了“半径”特征的可靠性。在第四章中,我们提出了一个可以利用“半径”特征的检测模型——SpamTracer。SpamTracer基于隐马尔可夫模型,我们将隐马尔可夫模型改造为了监督型分类模型。SpamTracer输入一个用户所有评论的按时间顺序排序的“半径”特征序列,根据训练数据分别计算该特征序列属于普通用户或水军用户的概率,最后比较概率大小,输出判断结果。我们在数学上推导了概率的计算过程,证明了模型的可实践性。在第五章中,我们利用提出的SpamTracer模型,进行了水军行动规律的探索。我们使用SpamTracer扩展了数据集,给所有未标签的评论都打了标签,并提出算法使用这些数据分析水军在时间上和地理位置上的行动规律。在第六章中,我们首先介绍了数据集的选择过程,以及最后采用的数据集的特点。我们使用的数据集是一个部分标签的Yelp数据集,包含店铺信息、评论信息和用户信息。我们从这个数据集里提取出我们需要的特征序列,将带标签的数据输入SpamTracer模型进行训练并测试准确率。为了体现出SpamTracer的优势,我们选择了四个经典分类算法——朴素贝叶斯分类器NB、支持向量机SVM、AdaBoost分类器和多层感知机MLP。我们以广泛使用的四项测试指标,精确度Precision、召回率Recall、准确率Accuracy和F1值,使用10层交叉验证方法对SpamTracer和对照组模型进行了测试。测试结果显示,在不平衡数据集中,SpamTracer的优秀的稳定性和超过平均水平的准确率获得了最好的表现。同时,我们也发现了SpamTracer仍然有进步空间,我们分析了SpamTracer模型的不足之处,提出了可能的限制因素。此外,我们还公布了我们关于水军行动规律分析的实验结果。我们发现水军门倾向于在一年中的夏季和一周中的周末进行频繁的集团水军活动,以及在店铺的开业初期进行水军活动以帮助新店造势。我们还发现了不同店铺间共有水军的数量与店铺的间隔距离成反比例关系,这说明小部分水军会在一个小商圈内与多家店铺进行合作。
