%# -*- coding: utf-8-unix -*-
\begin{thanks}

  水军检测这个课题的研究进行了近一年左右的时间。从课题的确定,到课题背景资料的扫盲,再到和指导老师及同组的同学各种探讨研究思路和实验方法,一路下来着实非常辛苦。但是在论文完成的那一刻,我感到这份辛苦是有价值的。比较遗憾的是,我们本来打算把研究成果写成一篇英文论文,并投稿到国际会议ASIACCS 2018上,但是很可惜没有被接收。虽然很难受,但是科研的道路注定是坎坷的,也不是所有人都是第一次写论文就能被如此高水平的国际会议接收的。我们还会就这个工作的成果继续投稿,接下来就一边努力写新的论文,并祈祷接下来的投稿能有好结果,一边准备前往异国他乡,继续踏上新的研究道路。
  
  在本次毕业设计完成之际,我首先要感谢论文指导老师——阮娜老师。从大二的暑假藉由PRP项目进入CSIC实验室以来,阮老师就在科研方面给予了我大量的帮助和指导。刚开始做的PRP项目是关于大型数据中心安全相关的研究,在半年多的研究中,我一边扫盲一边自学,并没有取得特别大的科研进展,看着同专业的同学们都在自己的科研之路上一路高歌猛进,我非常焦急。在这个关键的时刻,阮老师及时地将我从这个已经被其他研究者讨论的没什么剩余价值的课题中拉出来,并介绍给我了水军检测这一个十分有趣且研究前景良好的方向,而且把我和两个同样对水军检测方向感兴趣的同学组成了课题小组。在老师的指导下,我从一个人在陌生领域的摸爬滚打,到和同伴一起研究新颖有趣的课题。阮老师对学术研究风向的嗅觉十分敏锐,制定了将现在大热的人工智能技术和水军检测技术相结合的研究方向。而且每次研究陷入瓶颈时,阮老师总能一针见血地帮我们指出突破口。在撰写论文的时间里,阮老师对待学术的严谨态度,对研究热点的把握,对课题组的关怀,帮助我在良好的氛围中入门了科研之路,令我受益匪浅。在此向敬爱的老师致以最衷心的感谢。
  
  我还要感谢课题组的邓若愚同学和陆羽同学。一年间我们一起做水军检测课题,集思广益一起想点子,若愚的编程和陆羽关键时刻的灵光一闪是这个课题得以成功结题的关键。多次承蒙二位的帮助,我由衷地感谢你们,和你们这样优秀的同伴一起从事科研十分愉快。
  
  此外,我还要感谢远方的家人们。在大学期间,我的父母一直在远方给我各种各样的支持,而且让我自己选择了未来的方向,还无条件地支持我做出自己的选择,我对此深表感动。也感谢我的友人们,由于共同的兴趣爱好我们相遇,在学校和来自各个学院各个年级朋友们一起参加各种活动,在家里和一直保持联系的老朋友们一起聚会,你们为我的大学生活带来了色彩,感谢有你们陪伴的时光。
  
  最后,感谢上海交通大学。这里是一切的开始,能在这里度过我的大学生涯,是我当时做出的最正确的选择。谢谢。

\end{thanks}
